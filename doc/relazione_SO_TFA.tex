\documentclass[a4paper,12pt]{report}
 
\usepackage[italian]{babel} % Adatta LaTeX alle convenzioni 
\usepackage[T1]{fontenc} % Riga da togliere se si compila con PDFLaTeX
\usepackage[utf8]{inputenc} % Consente l'uso caratteri accentati italiani
\usepackage{listings}
\usepackage{blindtext}
\usepackage{courier}

\lstset{language=C}
 
\frenchspacing % forza LaTeX ad una spaziatura uniforme, invece di lasciare più spazio
% alla fine dei punti fermi come da convenzione inglese
 
\title{Sistema Operativo su Atmega328p (Piattaforma hardware usata Arduino UNO)} 
 
\author{Massi Gionata \\ TFA A042\\Pelliccioni Giovanni\\ TFA A034}
\date{8 aprile 2015}

\begin{document}
\maketitle 
\begin{abstract} 
Questo documento mostra come realizzare un Sistema Operativo su microcontrollore ATMega328p come argomento di studio e e realizzazione pratica nelle classi terminali di un Istituto Tecnico o di una Scuola Professionale ad indirizzo informatico. Lo scopo \'e quello di far comprendere agli studenti quali siano le componenti essenziali di un sistema operativo e come esse possano essere realizzate. Sebbene i concetti esposti siano complessi, gli studenti sono facilitati dalla presenza di un codice base eseguibile e potranno sperimentare la creazione di nuovi compiti o la modifica di parti del sistema. La strategia didattica presa a modello \'e quella del ``\textit{learning by doing}''.
\end{abstract} 
 
\tableofcontents % Prepara l'indice generale
 

%This document accompanies source code for a simple operating system for micro-controllers.  The OS uses cooperative multitasking and handles a few different tasks (but you can easily add more).


Questo documento illustra come realizzare un semplice sistema operativo per microcontrollori.
Il sistema operativo utilizza un multitasking cooperativo e gestisce alcuni compiti, permettendo diaggiungerne facilmente dei nuovi.

\chapter{Bathos e Thos}

Questo sistema viene ripreso da Bathos (Born-Again Two Hours Operating System) di Alessandro Rubini.  Bathos è uno spin-off di Thos (Two Hours Operating System) il sistema operativo che egli realizza ed espone ai sui studenti universitari in due ore. Per ovvie ragioni non \'e possibile replicare la presentazione a studenti delle scuole medie superiori negli stessi tempi e negli stessi modi, ma l'impianto del sistema \`e tale da essere compreso anche dal target selezionato.

Il presente lavoro include parte della documentazione di Thos, tradotta in italiano. Esso descrive la realizzazione di un intero sistema operativo ``giocattolo'' che Alessandro Rubini porta a compimento in due ore. Il codice sorgente che utilizziamo come base \'e quello di BA-Thos, che aggiunge funzionalit\'a a Thos, modificandone la struttura del codice, aggingendo la funzione \texttt{printf}, alcuni driver e altro ancora.
L'impianto didattico, scopo del progetto, non viene intaccato e pertanto il progetto si presta ai nostri scopi.

%Bathos is a spin-off of Thos, the Two-Hour Operating System.
%For that reason, the complete documentation of Thos is included in
%this manual: it describes how to write the whole OS in
%two hours.  However, Bathos is more than Thos, so later chapters
%change completely the code layout, to achieve a more portable
%and flexible design; they add \texttt{printf}, some drivers and other stuff.
%This without changing the basic design ideas of Thos, despite the change
%in code layout.


\section{Introduzione a Bathos}

Dopo aver mostrato il progetto Thos agli studenti, essi potranno usarlo per realizzare progetti reali, e ciò può anche permettere di approfondire tematiche specifiche attraverso la pratica del gioco, ovvero con la pratica del ``learning by doing''. Lo scopo di Bathos è quello di continuare lo sviluppo, l'aggiunta di una implementazione di \textit{printf}, così come alcuni driver di periferica. L'idea di base è comunque mantenere il codice più semplice possibile, evitando gli interrupt.

%After showing the Thos project to my students,
%several of them actually made real projects with it, and I found
%myself playing with Thos as well.  So finally I decided to continue
%development, adding a real \textit{printf} implementation (which I had
%around anyways), as well as some device drivers. The basic idea is still to
%keep the code as simple as possible and avoid interrupts. However, you
%can write real stuff with it.

%This document is unchanged in its first part, which remains the ``two
%hour'' idea.  Everything from  onwards is
%material I added later.  In order to look at the ``2-hour'' code, as
%described in the initial chapters of this document, you should check
%out commit , which is the commit before
%Thos becomes Bathos -- the first step is a massive rename.  As an
%alternative, you can checkout  or tag,
%which is the final release of Thos, with support for Arduino and references
%to this newer project. The current
%\textit{git} repository is rooted on the complete history of Thos
%up to .

% Lo vogliamo lasciare???
L'idea del documento è quella di fare un sistema operativo in `` due ore ''. L'attuale repository \textit{git} contiene tutto il progetto da Thos fino a Bathos. Al fine di analizzare il codice  in `` 2 ore '', come descritto nei capitoli iniziali di questo documento, si dovrebbe verificare il commit, in modo che Thos diventi Bathos - il primo passo è una ridenominazione di massa. La versione finale di Thos, contiene anche il supporto per Arduino e riferimenti di questo progetto. 

\section{Introduction to Thos}

La CPU di destinazione per Thos era il Cortex-M3 LPC1343. L'idea era quella di creare una \textit{Board Versatile} in modo di poter iniziare a far girare il codice su \textit{QEMU}, che è un virtualizer ed emulatore di macchina open source, senza un hardware reale.


I prerequisiti per seguire questo progetto e per la scrittura del sistema operativo sono la conoscenza del linguaggio C e un po di esperienza con la programmazione generale. 

Il package rilasciato è un albero git, in modo da poter tornare a versioni precedenti nella storia delle release pubblicate.



%The target CPU for Thos is the LPC1343 (Cortex-M3). The board I  used
%is the chap one by Olimex (which actually has even cheaper boards now,
%based on the same CPU); however, I had to add a level converter
%for the serial port.  Actually, any board with that CPU
%will work; unfortunately you, as a reader,
%are not actually expected to run the code because of lack of hardware,
%but please be assured that this code works fine on my hardware.
%
%In the future I may port the system to the \textit{Versatile} board, in
%order to be able to run it on \textit{Qemu}, without real hardware; Bathos,
%in fact, already runs on the \textit{Versatile}. No
%other changes to Thos are planned, because it must fit in two hours.
%(Once, in August 2012, I made an error with time slots and shrank
%my talk into one hour and 15 minutes, but it's not fun, that fast -- the
%video of that talk is available on the Internet).
%
%This project was born as a simplified spin-off of another OS that I am
%writing; Thos has been presented a few times in the form of a lesson.
%Initially I wrote it for ARM7 (using the \textit{Christmas Tree} device),
%then I've redone it on a Cortex-M3, since the \textit{tree} is not for sale
%any more and the cheap ARM7 boards are now more expensive than the
%cheap Cortex ones, so the current hardware I'm using is the board
%depicted below. 
%
%
%
% describes the whole writing of the OS, starting from an
%empty directory. As a prerequisite, you are expected to know C
%language and have some experience with computer programming in
%general.  Master or PhD students are usually able to understand and enjoy
%the Thos lesson.
%
%The released package is a git tree, so you can get back to older
%releases in the published history and follow the code described in the
%text as it is being written -- however, the document is committed
%after the code in the published history, so you need to keep a copy of
%the documentation before checking out individual sections.
%
%Please note that sometimes the choices being made are suboptimal, and
%I'm well aware of it. This material is really being shown in two
%hours, so many details have been ignored because of time constraints.
%I'm somehow shy of the suboptimal choices now that I put it all in
%writing, but I still want this material to be explained while being
%typed live in a single lesson.


\chapter{Writing Thos} 

I concetti spiegati qui vengono mantenuti, quasi invariati, in Bathos. Alcuni dettagli sono cambiati per ottenere una migliore portabilità.



\section{Laying out the Makefile} 


Come primo passo nella creazione di un nuovo pacchetto, verrà impostato un \textit{Makefile}, avendo cura di compilare i sorgenti che sono sia file C e che file assembly. Si userà un semplice \textit{Makefile} (il più semplice possibile).


\subsection{Cross-Compilation} 

Il codice verr\'a cross-compilato: il target \`e il microcontrollore ATmega328p, mentre il sistema di compilazione \`e composto da un PC. Si inizia definendo gli strumenti utilizzati per la cross-compilazione: la prassi consolidata è la definizione CC e delle altre variabili sulla base di una variabile d'ambiente chiamata \texttt{CROSS\_COMPILE}. Si  copia la solita stringa \textit{Makefile} per qualsiasi versione del kernel: 
%The code will be cross-compiled: the target is ARM while the build
%system is a PC. So let's start by defining the tools used for
%cross-compilation: the established practice is defining CC and
%other variables based on an environment variable called
%\texttt{CROSS\_COMPILE} .  Thus, let's copy the usual stanza from the
%\textit{Makefile} of any kernel version:


\begin{lstlisting}
   AS              = $(CROSS_COMPILE)as
   LD              = $(CROSS_COMPILE)ld
   CC              = $(CROSS_COMPILE)gcc
   CPP             = $(CC) -E
   AR              = $(CROSS_COMPILE)ar
   NM              = $(CROSS_COMPILE)nm
   STRIP           = $(CROSS_COMPILE)strip
   OBJCOPY         = $(CROSS_COMPILE)objcopy
   OBJDUMP         = $(CROSS_COMPILE)objdump
\end{lstlisting} 


\end{document}









%
%@example
%   AS              = $(CROSS_COMPILE)as
%   LD              = $(CROSS_COMPILE)ld
%   CC              = $(CROSS_COMPILE)gcc
%   CPP             = $(CC) -E
%   AR              = $(CROSS_COMPILE)ar
%   NM              = $(CROSS_COMPILE)nm
%   STRIP           = $(CROSS_COMPILE)strip
%   OBJCOPY         = $(CROSS_COMPILE)objcopy
%   OBJDUMP         = $(CROSS_COMPILE)objdump
%@end example
%
%The cross-compiler we need to use is any @i{gcc} starting from version 4.3. If
%your compiler is older than that, it won't support the @i{armv7},
%which is our target CPU.   The most common choice nowadays is using the
%@i{lite} version of @code{codesourcery.com}.  For example, you can
%download this file:
%
%@c http://www.codesourcery.com/sgpp/lite/arm/portal/package6493/public/
%@c      arm-none-eabi/arm-2010q1-188-arm-none-eabi-i686-pc-linux-gnu.tar.bz2
%@smallexample
%http://www.codesourcery.com/sgpp/lite/arm/portal/package8734/public/
%      arm-none-eabi/arm-2011.03-42-arm-none-eabi-i686-pc-linux-gnu.tar.bz2
%@end smallexample
%
%(you may look for a more recent version, but all of them are a dozen clicks
%away from the home page, so the direct link here may be useful).
%
%The compiler can be uncompressed  to any directory of your choice. For
%example, if you unpack in @code{/opt}, you'll need to run this command
%
%@example
%   export CROSS_COMPILE=/opt/arm-2011.03/bin/arm-none-eabi-
%@end example
%
%in order for the @i{Makefile} to build all names for your commands.
%
%@c --------------------------------------------------------------------------
%@node CFLAGS
%@subsection CFLAGS
%
%The Cortex-M3 is only one of the possible flavors of the ARM processor,
%so you'll need to always pass @code{-march=armv7-m -mthumb} to both
%the compiler and the assembler.  Thus, our @i{Makefile} will need
%the following two lines as well. The extra @code{-g} forces
%debug information to be generated, and @code{-Wall} forces all warnings
%to be reported; both are things you really can't live without:
%
%@example
%   CFLAGS  = -march=armv7-m -mthumb -g -Wall
%   ASFLAGS = -march=armv7-m -mthumb -g -Wall
%@end example
%
%With these variables in place, @i{make} is able to automatically create
%@i{file.o} if either @i{file.c} or @i{file.S} exist.   We only need to
%declare our list of object files and a rule to link them; let's use
%the handy @i{make} shortcuts that name the target and the dependents,
%to avoid repetition of file names:
%
%@example
%   thos: boot.o io.o main.o
%           $(LD) $(LDFLAGS) $^ -o $@
%@end example
%
%@c --------------------------------------------------------------------------
%@node A First Build
%@subsection A First Build
%
%We'll now ``@code{touch boot.S io.c main.c}'' to at least be able to
%start a compilation. The choice of the file names reflects the need to
%have at least some operation running, with the necessary boot steps
%and minimal output operations in place.
%
%With the @i{Makefile} and the empty files in place, we can issue
%@i{make}, which finally builds a binary with no actual
%contents, as all input object files are empty. It also
%spits a warning, explained in the next chapter:
%
%@smallexample
%   ld: warning: cannot find entry symbol _start; defaulting to 00008000
%@end smallexample
%
%The current code situation is committed in the repository, like I
%always do for every chapter and section of this document. The
%commit message is a one liner, naming the chapter (or section)
%where the code is introduced.
%
%@c ==========================================================================
%@node The Linker Script
%@section The Linker Script
%
%To perform the final link step, we actually need a @i{linker script}.
%In other words, we must tell the linker how to lay out the memory
%map for the program.
%
%@c --------------------------------------------------------------------------
%@node Spelling the Object Format
%@subsection Spelling the Object Format
%
%When compiling your @code{hello.c} or equivalent program, you'll end
%up using the default linker script (usually in
%@i{/usr/lib/ldscripts}), which is designed to take care of programs
%that are hosted within an operating system.  When writing Thos we need to
%provide our own linker script, because our program must reside
%directly in the memory of our micro-controller and can't count on an
%host operating system.
%
%The first three lines of our linker script are standard stuff, that
%I refuse to learn by memory and I simply copy over from another linker
%script; the only customized line here is the entry point, which 
%we want to be @code{_thos_start}, a @i{symbol} that we'll define
%in our small assembly source file.
%
%@example
%   OUTPUT_FORMAT("elf32-littlearm")
%   OUTPUT_ARCH(arm)
%   ENTRY(_thos_start)
%@end example
%
%@c --------------------------------------------------------------------------
%@node Placing Sections in Memory
%@subsection Placing Sections in Memory
%
%After the header part, the script must list the ELF output sections,
%and the address in memory where they are placed.  To this aim, we need
%to know the memory map for the target processor.
%
%In the specific case (LPC1343), the processor has internal RAM and
%internal Flash; a user binary can be programmed to Flash memory using
%the USB port, or can be programmed to either RAM or Flash using a
%serial port.  Since programming to RAM is easier (both the associated
%@i{linker script} and the startup code are easier), my choice here is
%programming to RAM, even if the serial port is not directly wired out
%in the evaluation board as sold.
%
%This means you'll have to connect a serial port in some way to test the
%code shown here: the UART is really the easiest way to print some
%diagnostic message.  This is unfortunate, but doing it differently
%would require a huge lot of time (remember, this is meant to happen in
%a lesson of two hours in total).  Actually, the board where I initially
%wrote THOS had a usb-serial interface, so there it was not an issue.
%
%If you get a copy of the LPC1343 manual, you'll find that RAM starts
%from address 0x1000.0000, and the first bytes of the RAM itself are
%used by the internal ROM during the loading procedure, so I chose
%to load the program 1kB within RAM -- later, I'm going to use that
%spare kB for the stack.
%
%The resulting linker script is as shown below. Note that @i{dot}
%(@code{"."})  is the point where output data is being written; thus,
%the initial assignment to @i{dot} forces the output to be placed at
%address 0x1000.0400.   Starting at that address, the usual ELF
%sections are placed, in the usual order. The sections we need
%are the old known @code{.code}, @code{.data} and @code{.bss}, with
%the more recent addition of @code{.rodata}, where the compiler
%places constant string and similar material.
%
%@example
%   SECTIONS
%   {
%           . = 0x10000400;
%           .text : {
%                   *(.boot)
%                   *(.text)
%           }
%           .rodata : { *(.rodata) }
%           .data : { *(.data) }
%
%           .bss : {
%                   . = ALIGN(16);
%                   __bss_start = .;
%                   *(.bss);
%                   . = ALIGN(16);
%                   __bss_end = .;
%           }
%   }
%@end example
%
%The only unusual detail in the script just shown, is the
%@code{*(.boot)} line.  It says that the @code{.text} output section
%is built to include the contents of any @code{.boot} section appearing in
%input files followed by the contents of any @code{.text} input sections.
%This is how I force the contents of @i{boot.S} to be at the
%head of the output file.
%
%The leading @code{*} in the definition of input sections means
%``all input files'', but you could also
%write @code{start.o(.text)}.  This is a matter of personal preference:
%I prefer to define initial code by putting it in a specially-named
%section instead of hardwiring the input file name in the linker script.
%
%
%@c --------------------------------------------------------------------------
%@node Using the Script
%@subsection Using the Script
%
%To put our linker script in action we need to set @code{LDFLAGS} in 
%the @i{Makefile}:
%
%@example
%   LDFLAGS = -T thos.lds
%@end example
%
%Now compilation of the empty input ends with the following message,
%which confirms the linker script is being used:
%
%@smallexample
%   ld: warning: cannot find entry symbol _thos_start; defaulting to 10000400
%@end smallexample
%
%
%@c ==========================================================================
%@node The Assembly Code
%@section The Assembly Code
%
%When the system begins running at power-on, there is no pre-built
%context at all.  The only thing you can count on is the initial execution
%address, and you can place your assembly instructions there.
%
%@c --------------------------------------------------------------------------
%@node Naming an ELF Section
%@subsection Naming an ELF Section
%
%Before you are able to run code in a higher level language at system
%boot, you need to build your so-called @i{run-time environment}.
%Since C is a very low-level language (they say it's powerful like
%assembly and simple like assembly) you need two very simple things: a
%stack pointer and a zeroed BSS. Remember that the @code{.bss} section
%is allocated but not stored in the ELF binary: it must be zeroed at
%run-time.
%
%Our first source code will be the minimal amount of assembly needed
%to perform those initial steps.  We place the stack pointer before
%code (so the stack can grow in the kilobyte of space we left free
%at the beginning of RAM), and we zero the @i{.bss} ELF section by
%means of the symbols we defined in the linker script: @i{__bss_start} and
%@i{__bss_end}.
%
%The code in @i{boot.S} is going to live in the section called @i{.boot},
%so the linker script will place it before the default code section.
%This ensures that the first instructions in the binary file come
%from @i{boot.S}.
%
%The first line of the file is thus as follows:
%
%@example
%   .section .boot, "ax"
%@end example
%
%While ELF section names are arbitrary, it's an established convention
%to use names with a leading dot. Assembly directives have always been
%starting with dot, like @i{.section} above, so a casual reader can see
%at a glance whether a word unknown to her is an instruction or a
%directive.  In the old ages, section names were fixed, and the
%directives in assembly files were just @i{.text}, @i{.data} and
%@i{.bss}; this convention of using dot-prefixed section names remained
%when arbitrary names were introduced.
%
%The @code{"ax"} above means @i{allocate} and @i{executable}. The
%former flag is now implicit, but usually retained for compatibility
%with older tools; the latter flag is needed so ELF-reading tools know
%what the content is.  In this case, we'll need to disassemble with
%@code{objdump -d}, and the tool refuses to disassemble data sections.
%In other execution environments (e.g., when the code is hosted in
%another OS) the flag may be used to set permissions on memory pages,
%so we may say that marking executable sections with @code{"x"} is
%mandatory, even if sometimes things may work nonetheless.
%
%@c --------------------------------------------------------------------------
%@node Actual ASM code
%@subsection Actual ASM code
%
%Within the @i{.boot} section we'll now place our code, labelling the
%first instruction as @code{_thos_start}, which is the @i{entry point}
%declared in the linker script.  Even though the entry point is not
%actually needed in this project, by defining it we avoid the ugly warning
%message; again, in other execution environments defining the entry
%point is mandatory, especially when it doesn't mark the first byte of the
%binary file.
%
%The leading underscore in the name of the entry point is
%another convention: symbols that are used in assembly files or other
%low-level implementations should not be referenced by higher-level
%code: the underscore is a signal that the symbol is somehow special
%and it's only relevant to people who hack the internals.
%
%The code then will set the stack pointer and zero bss. The stack
%address is calculated by subtracting from the program counter, to save
%an explicit address in the code, the bss-zeroing loop is then
%trivial. You are not expected to be able to write assembly code for
%your target CPU, but you are always expected to be able to read it;
%fortunately all assemblers are similar, so this is rarely a problem.
%
%@example
%   .global _thos_start
%   .extern __bss_start
%   .extern __bss_end
%   .extern thos_setup
%   .extern thos_main
%
%   _thos_start:
%   /* set the stack pointer */
%           mov r0, pc
%           sub r0, #12
%           mov sp, r0
%
%   /* clear BSS */
%           mov r0, #0
%           ldr r1, =__bss_start
%           ldr r2, =__bss_end
%   0:      
%           cmp r1, r2
%           bge 1f
%           str r0, [r1]
%           add r1, #4
%           b 0b
%
%   1:      bl thos_setup
%           bl thos_main
%           b 1b
%@end example
%
%There are a few details worth explaining here, which are independent
%of the target architecture and assembly language:
%
%@table @i
%
%@item .global
%
%	Any symbols defined in the assembly file are local by default.
%        Here, we want @i{_thos_start} to be externally visible, because
%        it is the entry point looked for by the linker script.
%
%@item .extern
%
%	Any undefined symbol is considered @i{extern} by default,
%        so all these @i{.extern} directives are actually optional.
%        However, stating that we know these symbols are known to be
%        missing from this very file makes it more readable.
%
%@item local labels
%
%	In Unix assemblers, and numeric-only label is considered local,
%        and is referenced like shown above: @code{0b} means ``0 backward''
%        and @code{1f} means ``1 forward'' -- the nearest local label
%        forward or backward from the instruction naming it.
%        This means that you can use the same label name several times
%        in the same source file, which is very useful when you write
%        assembly code in C macros.
%
%@end table
%
%Finally, the magic @code{#12} constant is there for a strange reason,
%as any constant will actually work (even 0).  In ARM, when an
%instruction reads the program counter, the value returned points two
%instructions after the current one.  Here, the PC is read in the first
%machine instruction ever, so it points to the third instruction.  If
%we want the stack to live before the code, we want to subtract three
%instruction lengths.  In @i{thumb} (our case), instructions are
%2-bytes long, but full ARM has 4-byte instructions, so 12 works in
%every case.  Just copying the PC to SP will work as well, but you'll
%overwrite the first two instructions of your program with your stack.
%
%@c --------------------------------------------------------------------------
%@node Compiling boot.S
%@subsection Compiling boot.S
%
%With the assembly file in place, compilation now fails with
%@code{undefined reference to `thos_main'} and @code{undefined
%reference to `thos_setup'}. That's expected: the two @code{.extern}
%symbols that are marked as @i{undefined} in @code{start.o} are not
%provided by other object files.  The error message we get back
%reports the exact source line where the undefined reference is
%found, this only happens because we used @code{-g} in
%compilation flags.
%
%To quickly fix two undefined symbols, let's simply define two
%empty functions, @i{thos_setup} in @code{io.c} and @i{thos_main} in
%@code{main.c}, as that's the place they'll finally live:
%
%@example
%   int thos_main(void)
%   { return 0; }
%@end example
%
%The function returns an integer value even if at this point we know
%the caller is ignoring such return value. In general, it's good
%practice to always return an error code (instead of @code{void}) even
%in simple functions, because otherwise you risk to change it later, when
%complexity increases as development goes on.
%
%At the end of this chapter we have a complete binary, which does nothing
%but at least compiles and has some real content (here and elsewhere,
%@code{morgana%} is the command prompt of my host):
%
%@example
%   morgana% nm thos | sort
%   10000400 T _thos_start
%   10000428 T thos_setup
%   10000438 T thos_main
%   10000450 B __bss_end
%   10000450 B __bss_start
%@end example
%
%@c ==========================================================================
%@node Loading the Binary
%@section Loading the Binary
%
%With a ``working'' binary in our hands, it's now high time to write it
%to the final hardware.  The preferred way to do that is using the
%serial protocol offered by the internal ROM in the CPU.
%
%@c --------------------------------------------------------------------------
%@node Pin-strapping
%@subsection Pin-strapping
%
%When power is applied to the LPC1343, pin P0_1 selects whether the CPU
%executes the program stored in internal flash or it runs code stored
%in internal ROM.  Such ROM can handle re-programming flash memory
%using the USB interface, or interact through the serial port.  Such
%behavior is selected with pin P0_3: if it is low, the serial port is
%used.
%
%Thus, in my board I added a jumper on P0_3 to force serial
%communication (the jumper on P0_1 was already provided by the
%manufacturer).  With both P0_1 and P0_3 pulled low, you can write to
%RAM or flash using the serial port.  The subdirectory @i{tools} of
%this package includes two programs: @code{program} and @code{progrom},
%they write a binary to either RAM or flash (here called ROM for
%symmetry).
%
%@c --------------------------------------------------------------------------
%@node Creating a Binary File
%@subsection Creating a Binary File
%
%The binary that must be programmed to the CPU, however, is not the ELF
%file, but a pure binary file, with no header or other extra information.
%To turn our ELF file into a binary we can use @i{objcopy}, with the
%following rule in @i{Makefile}:
%
%@example
%   thos.bin: thos
%           $(OBJCOPY) -O binary $^ $@
%@end example
%
%Note that it is possible for a loading program to directly use the ELF
%file: writing an ELF loader from scratch is a matter of a few hours,
%if you have previous experience half an hour may be more than enough.
%Most other tools, though, load a bare binary file, so I chose to do
%the same here; moreover, by having a binary file on disk you can
%easily compare it with a dump of your flash memory. For disassembly
%we'll still need the ELF file, to see symbolic names in the code.
%
%@c --------------------------------------------------------------------------
%@node UART Programming
%@subsection UART Programming
%
%The tools can be compiled by running @code{make -C tools}, as I lazily
%won't be changing our @i{Makefile} to build them.  Finally,
%programming with the serial port and @i{tools/program} looks like the
%following (where @code{ostro%} is my shell prompt):
%
%@example
%   ostro% ./tools/program thos.bin
%   Opening serial port /dev/ttyUSB0
%   Forcing boot loader mode
%   Syncronizing... done
%   Identifying... done
%   part number: 3d00002b
%   LPC1343, 32kB Flash, 8kB RAM
%   size is 900
%   W 268436480 900
%   0
%   ....................OK
%@end example
%
%After this, the program mirrors data from @i{stdin} to the serial port
%and from the serial port to @i{stdout}. This will be useful to see the
%output messages of @i{thos}, but at this point there is nothing to
%look at.
%
%@c ==========================================================================
%@node Serial Output
%@section Serial Output
%
%One advantage we achieve as a side effect of using the UART port for
%programming, is that the serial port has already been configured for
%115200 baud.  Thus, the first snippet of real code in the OS will be
%some output function.  Let's write a simple @i{putc} and @i{puts},
%both in the @code{io.c} file.
%
%@c --------------------------------------------------------------------------
%@node puts
%@subsection puts
%
%The implementation of @i{puts} is trivial: it just calls
%@i{putc} for each byte until end of string:
%
%@example
%   void puts(char *s)
%   {
%           while (*s)
%                   putc (*s++);
%   }
%@end example
%
%@c --------------------------------------------------------------------------
%@node An I/O Model
%@subsection An I/O Model
%
%As for @i{putc}, the function must write the next character to the TX
%register in the hardware, but only when there is no outstanding
%transmission, or the byte will be lost.  By looking at the CPU manual,
%we'll easily find the @code{U0THR} register (UART 0 Transmit Holding
%Register) and the @code{U0LSR} register, with its @code{THRE} bit
%(Line Status Register, Transmit Holding Register Empty).
%
%Once the register values are known, the code is pretty simple, however,
%we need to define our I/O primitives: a standardized way to access
%hardware registers.  Here a few approaches are possible. Let's list
%a few possibilities
%
%@table @i
%@item Defining register names like they were variables
%	This approach is pretty common in the small microcontroller-class
%        operating system. The resulting code to write the Transmit Holding
%        Register is somewhat like ``@code{U0THR = c}''.  I personally
%        dislike it because the casual reader should know in advance
%        that @code{U0THR} is an hardware register, and hardware accesses
%        are not generally visible in the code.
%
%@item Using @code{readl} and @code{writel} helpers
%	This is the Linux approach.  Every access to registers is performed
%        by pointer and is consistently encapsulated in a special function.
%        It is a very good option, and one worth following; it has the
%        minor disadvantage that readers are expected to know the convention.
%
%@item Using a @code{regs} array to address register
%	It is an alternative way to mark register access. While less
%        widespread than the @i{readl}/@i{writel} helpers, I enjoy the
%        fact it's immediately understood by the casual reader; moreover
%        it helps perusing the linker script, which I find useful for
%        teaching purposes.
%
%@end table
%
%This project, therefore, uses @code{regs} as an array of registers.
%The array is defined extern in a new header, @code{hw.h}, and
%exploits the fact that all registers are 32 bits wide; finally,
%is uses the standard sized type @code{uint32_t} from @code{<stdint.h>}:
%
%@example
%   #include <stdint.h>
%   extern volatile uint32_t regs[];
%@end example
%
%Here, the @code{volatile} attribute is needed to kill compiler
%optimizations over read/write operations concerning the array: every
%read or write operation that appears in source code will actually
%perform a read or write instruction in generated machine code.
%Clearly, also the other I/O approaches listed above hide a
%@code{volatile} keyword somewhere in their implementation.
%
%@c --------------------------------------------------------------------------
%@node Defining Registers
%@subsection Defining Registers
%
%With @code{regs} properly declared, we need to define register
%names for our needs.  Since the @code{U0THR} register lives at
%address 0x40008000 we'll want that exact number to appear in our
%@code{hw.h}, to ease people grepping for the symbolic name in the
%source code.  The easiest approach in this case is arranging for
%@code{regs} to be 0; the actual registers and bits we need are then
%defined in this way:
%
%@example
%   #define REG_U0THR               (0x40008000 / 4)
%   #define REG_U0LSR               (0x40008014 / 4)
%   #define REG_U0LSR_THRE          0x20
%@end example
%
%Actually, an extra-abstract implementation will use
%@code{sizeof(regs[0])} in place of the explicit 4, but here I prefer
%to assume people knows 32-bit registers are 4-byte long.
%
%The chosen implementation is only one of a number of options, but it
%has a few good points: it preserves the hex number that you find in
%the CPU documentation, re-stresses that such things are registers by
%using @code{REG_} as a prefix, and finally enumerates bits in the name
%space of the hosting register by using a second underscore.
%
%@c --------------------------------------------------------------------------
%@node putc
%@subsection putc
%
%With the three definitions in place, our @i{putc} turns out to be
%like this:
%
%@example
%   void putc(int c)
%   {
%           if (c == '\n')
%                   putc('\r');
%           while ( !(regs[REG_U0LSR] & REG_U0LSR_THRE) )
%                   ;
%           regs[REG_U0THR] = c;
%   }
%@end example
%
%The function as shown takes also care of newlines: whereas C code uses
%@i{newline} alone to mark newlines, the UART conventions want
%@i{return-newline}, so the fix is performed at output time, to save
%source code from this boring detail.
%
%@c --------------------------------------------------------------------------
%@node The regs Array
%@subsection The regs Array
%
%The last missing point here is the definition of @code{regs}. This
%is simply done in the linker script, by means of the following line:
%
%@example
%   regs = 0;
%@end example
%
%This is no different from the definition of @code{__bss_start} and
%@code{__bss_end}; the linker script is actually there exactly for this
%reason: defining symbols -- in addition to placing code in memory.
%
%@c --------------------------------------------------------------------------
%@node Header and more CFLAGS
%@subsection Header and more CFLAGS
%
%To close up this chapter with a working binary, we need a few more
%details, that deserve no discussion, being pretty trivial:
%
%@itemize
%@item Create @code{thos.h} with prototype for all functions.
%@item Include @code{thos.h} in all C sources, including @code{io.c} even
%      if it doesn't call any external functions, to ensure the prototype
%      matches actual code.
%@item Call @i{puts} from @code{main.c}, to exercise the code.
%@item Add @code{-ffreestanding -O2} to @code{CFLAGS} in @i{Makefile}.
%@end itemize
%
%Optimizations are generally needed to make the code better; with the
%current code the overall binary size is reduced from 240 to 176 bytes
%(although actual figures may differ according to the compiler being
%used).
%
%The @i{freestanding} flag is need to avoid the following warnings:
%
%@example
%   thos.h:5: warning: conflicting types for built-in function 'putc'
%   thos.h:6: warning: conflicting types for built-in function 'puts'
%@end example
%
%The compiler knows what are the prototypes for a number of standard-compliant
%functions, so we must clearly state that we are not hosted in a Posix
%environment, thus the @code{-ffreestanding}.
%
%@c --------------------------------------------------------------------------
%@node Testing puts
%@subsection Testing puts
%
%With the code for this chapter, after programming we get an
%endless stream of:
%
%@example
%   The mighty Thos is alive
%@end example
%
%The @i{endless} part depends on the fact that @i{thos_main} returns, and
%our assembly code restarts the system after calling @i{thos_main}.
%
%@c ==========================================================================
%@node The Jiffies Variable
%@section The Jiffies Variable
%
%The system is now equipped with serial output, which is a great first
%step. We now need a way to handle timing.
%
%@c --------------------------------------------------------------------------
%@node The Timer Tick
%@subsection The Timer Tick
%
%The easiest way to handle time is having an integer variable that
%counts timer interrupts. For example once every 10ms.  As a matter of
%facts, the Cortex family of processors has a specific timer-tick
%interrupt line designed for this specific aim.
%
%Unfortunately, handling interrupts is not feasible in a single lesson,
%so we must look for alternatives.  While reading the CPU manual, a
%programmer might notice that the timers of the LPC1343 have a
%32-bit-wide prescaler; this means that the counter register itself can
%count as slowly as needed. For example it can count at 100HZ even if
%the input quartz we are using clocks the system as 12MHz.
%
%@c --------------------------------------------------------------------------
%@node Power-on the Timer
%@subsection Power-on the Timer
%
%In this specific SoC, when the device is turned on most peripherals are
%powered off until we gate their clock (the UART was already turned on
%only because we used the serial port for programming: the internal ROM
%turned it on and configured it for us).
%
%The first step before doing anything with the code is thus defining
%the registers and bits to configure the timer; let's turn on
%timer 1 (I plan to use timer 0 for an external buzzer I connected
%to pin 0_11):
%
%This is the definitions for the needed registers, preserving the names
%found in the CPU manual:
%
%@example
%   /* clock control */
%   #define REG_AHBCLKCTRL          (0x40048080 / 4)
%   #define REG_AHBCLKCTRL_CT32B0   (1 << 9)
%   #define REG_AHBCLKCTRL_CT32B1   (1 << 10)
%
%   /* counter 1 */
%   #define REG_TMR32B1TCR          (0x40018004 / 4)
%   #define REG_TMR32B1TC           (0x40018008 / 4)
%   #define REG_TMR32B1PR           (0x4001800c / 4)
%@end example
%
%@c --------------------------------------------------------------------------
%@node The HZ Macro and thos_setup
%@subsection The HZ Macro and thos_setup
%
%To make thing a little abstract, let's also define our frequencies
%in @code{hw.h}, so they can be readily changed when porting to
%a different board using the same CPU:
%
%@example
%   #define THOS_QUARTZ (12 * 1000 * 1000)
%   #define HZ 100
%@end example
%
%The explicit multiplication has been chosen for readability: the
%calculation is performed at compile time and you see at first sight
%that it is 12 millions -- on the other hand, a row of 6 zeroes is
%confusing and you'll find yourself checking it over and over if you
%encounter a bug related to timing.
%
%The choice of @code{HZ} as a short name for our ticking frequency
%is mimicking Linux, which is always a good choice given the huge amount
%of programmers already used to its conventions.
%
%With those additions to our headers, we are ready to fill
%@code{thos_setup}, the function we already have, though empty, in
%@code{io.c}:
%
%@example
%   int thos_setup(void)
%   {
%           regs[REG_AHBCLKCTRL] |= REG_AHBCLKCTRL_CT32B1;
%
%           /* enable timer 1, and count at HZ Hz (currently 100) */
%           regs[REG_TMR32B1TCR] = 1;
%           regs[REG_TMR32B1PR] = (THOS_QUARTZ / HZ) - 1;
%           return 0;
%   }
%@end example
%
%@c --------------------------------------------------------------------------
%@node Defining jiffies
%@subsection Defining jiffies
%
%@i{jiffies} is now counting ad 100Hz. We only need to see it from C
%code: following Linux tradition, let's declare it in @code{thos.h} and
%define it in the linker script, with the following two lines, one per
%file:
%
%@example
%   extern volatile unsigned long jiffies;
%
%   jiffies = 0x40018008;
%@end example
%
%Note that we might have defined @i{jiffies} in terms of @code{regs},
%without using the linker script again, or we could have used the name
%@code{REG_TMR32B1TC} by using the C preprocessor over the linker script.
%There are always several ways to achieve the same result, and you are
%free to change to code to test different implementations, this is only
%my personal choice for the sake of simplicity.
%
%@c --------------------------------------------------------------------------
%@node Using Our Time Facility
%@subsection Using Our Time Facility
%
%As a final step in this chapter , let's now exercise the @i{jiffies}
%variable: our main function can print its self-promotion string once
%per second:
%
%@example
%   int thos_main(void)
%   {
%           unsigned long j = jiffies;
%           while (1) {
%                   puts("The mighty Thos is alive\n");
%                   j += HZ;
%                   while (jiffies < j)
%                           ;
%           }
%   }  
%@end example
%
%The code, simple as it is, works perfectly once programmed to
%the target CPU.
%
%@c ==========================================================================
%@node The Task Model
%@section The Task Model
%
%The OS is now in good shape. It has messaging and a time source; it
%only needs a task model. To avoid introducing interrupts, which
%wouldn't fit in the available time, let's use cooperative multi-asking:
%each @i{task} is implemented by time-based @i{jobs} (the names being
%used here are common the real-time world).  A job is simply a function
%that returns when done.
%
%@c --------------------------------------------------------------------------
%@node Defining a Task Structure
%@subsection Defining a Task Structure
%
%Considering that most tasks you need to accomplish can be reasonably
%modelled as a state machine, the job function can receive its current
%state as an argument and pass the next state as return value to the
%caller.  The most generic state is a @code{void *}, because it can
%either be casted as a simple number or used to point to a more complex
%structure.
%
%The data structure we need is therefore as follows, in @code{thos.h}:
%
%@example
%   struct thos_task {
%           char *name;
%           void *(*job)(void *);
%           int (*init)(void *);
%           void *arg;
%           unsigned long period;
%           unsigned long release;
%   };
%@end example
%
%The @code{name} field is there for informational purposes, @code{job}
%is the state-machine and @code{init} allows to decouple any setting up
%of the hardware.
%
%The @code{init} function receives the same argument as the job itself
%so it can behave differently according to its context, the same effect
%can be achieved by passing a pointer to @code{struct task} (i.e. to
%itself). I discourage from being lazy in parameter passing, especially
%for functions that are not in the critical execution path.  For the
%same reason, @code{init} returns @code{int}, so the system may be
%extended to handle initialization failures, should the need arise.
%
%The @code{release} field is the release time for the job: it is
%initialised to the first activation time and then modified by the
%scheduler according to @code{period}, each time the job runs.
%
%@c --------------------------------------------------------------------------
%@node Stuffing Tasks in an ELF Section
%@subsection Stuffing Tasks in an ELF Section
%
%The @i{thos} operating system is one of the conventional real-time
%operating systems that are built as a single binary images, which
%includes all compile-time defined tasks in a single blob.
%
%The easy way to arrange compile-time static sets is by building
%arrays.  Instead of making an array for a predefined maximum number of
%tasks, we'd rather make an array of just the right size.  The best
%way to do this is by juxtaposing data structures and bless the
%boundaries of the data area with two symbolic names.
%
%By building a @code{.task} ELF section we can easily achieve that.
%The @i{gcc} compiler offers the @code{__attribute__} construct to
%assign code or data to specific sections, overriding the default one
%(@code{.text} for functions and @code{.data} or @code{.bss} for variables);
%
%To hide the hairy @code{__attribute__} syntax from the user, let's add
%this definition in @code{thos.h}:
%
%@example
%   #define __task __attribute__((section(".task"),__used__))
%@end example
%
%We need @code{__used__} in order to be able to declare the structure
%as @code{static} within the file where it is defined.  As a side
%effect, if we forget to spell out our @code{__task} macro the compiler
%will warn about a static structure being defined but not used.
%
%The linker script must collate all input @code{.task} sections into
%an output section, adding symbolic names in the usual form at the
%boundaries of such sections:
%
%@example
%   __task_begin = .;
%   .task : {*(.task) }
%   __task_end = .;
%@end example
%
%Finally, the two extern symbols are defined in @code{thos.h}:
%
%@example
%   extern struct thos_task __task_begin[], __task_end[];
%@end example
%
%@c --------------------------------------------------------------------------
%@node Using Wildcards in Makefile
%@subsection Using Wildcards in Makefile
%
%I am a lazy typist, and I get upset whenever i need to edit a file to
%activate a feature; so, I want my users to be able to just drop-in new
%tasks in the @i{thos} directory and have them run, without manually
%editing a task list.  The ELF section helped, but there is still the
%@code{Makefile} to be edited.
%
%In a burst of extra-laziness, let's use wildcards in the @code{Makefile}:
%
%@example
%    TSRC = $(wildcard task-*.c)
%    TOBJ = $(TSRC:.c=.o)
%
%    thos: boot.o io.o main.o $(TOBJ)
%@end example
%
%This is a bad idea in many situations, because whenever you make a
%backup copy of a source file in the same directory, you'll find that
%both files get compiled and the linker complains about duplicate
%symbols.  However, I find it handy in small projects, especially
%because I can drop-in a task that plays music without any edit
%-- this will happen in the next release of this document.
%
%The commit for this chapter also includes a @code{make clean} target,
%which is useful in a system with no support for real dependencies.
%
%@c ==========================================================================@
%node Three Simple Tasks
%@subsection Three Simple Tasks
%
%Let's write a few tasks that print strings on a timely basis. We can
%use a single job function, with a different argument. The source
%file is called @code{task-uart.c}:
%
%@example
%   #include "thos.h"
%   #include "hw.h"
%
%   static void *uart_out(void *arg)
%   {
%   	char *s = arg;
%   	puts(s);
%   	return arg;
%   }
%
%   static struct thos_task __task t_quarter = {
%   	.name = "quarter", .period = HZ/4,
%   	.job = uart_out, .arg = "."
%   };
%
%   static struct thos_task __task t_second = {
%   	.name = "second", .period = HZ,
%   	.job = uart_out, .arg = "S",
%   	.release = 1,
%   };
%@end example
%
%The actual file includes two more tasks, one running every 10s and one
%running every minute. There's nothing strange in these few lines, with
%the exception of the @code{release} field.
%
%The @code{release} field states when the task's job must be first
%invoked; if several jobs are released at the same time, the scheduler
%will have to choose the order of activation. By forcing a different
%activation time for each task we can release the scheduler from
%working on priorities.  Once again, the choice is dictated by the lack
%of time -- the two hours are almost over, so we must hurry to meet the
%deadline.
%
%@c --------------------------------------------------------------------------
%@node Writing the Scheduler
%@subsection Writing the Scheduler
%
%The scheduler is really simple: it must select which task is next to run,
%wait for its activation time to arrive, and call it. We put it inside
%@code{main.c}, after the @i{mighty} hello message:
%
%@example
%   while (1) {
%           struct thos_task *t, *p;
%
%           for (t = p = __task_begin; p < __task_end; p++)
%                   if (p->release < t->release)
%                           t = p;
%           while ((signed)(t->release - jiffies) > 0)
%                   ;
%           t->arg = t->job(t->arg);
%           t->release += t->period;
%   }
%@end example
%
%In the trivial code above, @code{t} is the selected task, and @code{p}
%is a pointer that scans the array.
%
%The @i{jiffies} comparison needs a cast to @code{signed} because the
%values are unsigned: without the cast any value different from 0 would
%be considered greater than zero.  Note that in Linux this detail is
%hidden in a @code{time_before()} macro, that also makes the code more
%readable. Had we used @code{jiffies < t->release} the code would have
%failed after 497 days; not a real problem here, but it's a good habit
%to always deal with overflows.
%
%@c --------------------------------------------------------------------------
%@node Task-set Initialization
%@subsection Task-set Initialization
%
%Something is missing here in the code just shown: we have no idea
%about what the initial value of @i{jiffies} is, and we are not allowed
%to write it (even if it was a software counter, modifying a variable
%that is being read by several users is bad practice and should be
%avoided as much as possible).
%
%The solution is simple: just add ``@code{jiffies + 2}'' to all
%activation times before using them. Actually, we take a snapshot
%of @i{jiffies} in a new variable, called @code{now}, and then we
%add ``@code{now + 2}'', because @i{jiffies} may have been incremented
%between two reads:
%
%@example
%   now = jiffies;
%   for (p = __task_begin; p < __task_end; p++)
%           p->release += now + 2;
%@end example
%
%We could add ``@code{now}'', but then the first activation time
%will already have expired, and we'd have wrong timing at each boot; not
%a real issue, but it is easily avoided.  We could add ``@code{now + 1}'',
%which is expected to be in the future, but is may be in the past by the
%time we completed the loop.  Using @code{now + 2} is surely in the future,
%even if @i{jiffies} is incremented immediately after we took our
%snapshot into @code{now}: the loop takes less than 10ms in a 12MHz CPU,
%however long the task list may be.
%
%Finally, let's use the @code{name} field in the task list, this is a good
%diagnostic aid to verify that everything went in place despite use of
%an obscure ELF section. In this same loop we'll also call the @i{init}
%method of each task, if assigned:
%
%@example
%   for (p = __task_begin; p < __task_end; p++) {
%           puts("Task: "); puts(p->name); putc('\n');
%           if (p->init) p->init(p->arg);
%   }
%@end example
%
%This loop must run before the @code{release} times are updated,
%because UART output is slow: at 115200 baud we have output only 11
%bytes per millisecond.
%
%@c --------------------------------------------------------------------------
%@node Running the Task Set
%@subsection Running the Task Set
%
%We are really done: let's just run @code{./tools/program thos.bin}
%to see that things do really work:
%
%@example
%   [...]
%   ....................OK
%   The mighty Thos is alive
%   Task: minute
%   Task: 10second
%   Task: second
%   Task: quarter
%   .S
%   minute!
%   ....S....S....S....S....S....S....S....S....S....S
%   ....S....S....S....S....S....S....S....S....S....S
%   ....S....S....S....S....S....S....S....S....S....S
%   ....S....S....S....S....S....S....S....S....S....S
%   ....S....S....S....S....S....S....S....S....S....S
%   ....S....S....S....S....S....S....S....S....S....S
%   minute!
%   ....S....S....S....S....S....S....S....S....S....S
%   ....S....S....S....S....S....S....S....S....S....S
%   ....S....S....S....S....S....S....S....S....S....S
%@end example
%
%@c ==========================================================================
%@node GPIO Output
%@section GPIO Output
%
%My major disappointment with the current code is that the OS provides
%for tasks as state machines, but the feature is not being used. There
%is also support for an @i{init} method, but nobody is using it.
%
%This chapter adds a new task, one that flips 4 bits, each connected to
%a LED. Bits 0,1,2,3 of GPIO port 3 are connected to leds, so
%they are perfect for us.
%
%By checking the CPU manual we find we need two new registers in
%@code{hw.h}, shown below, and one bit in the control register
%(@code{REG_AHBCLKCTRL_GPIO}), not shown here:
%
%@example
%   #define REG_GPIO3DAT            (0x50033ffc / 4)
%   #define REG_GPIO3DIR            (0x50038000 / 4)
%@end example
%
%The source file is called @code{task-led.c} and no other file
%is changed, as we have wildcard support in @code{Makefile}.
%The job runs 5 times per second and features a running light
%with a period of 1 second: 4 states turn on one led, in sequence,
%and the last state has all leds turned off.
%
%My initial example was shorter (used 4 states, so no @code{if}
%nor @code{switch} was needed) but this is more realistic as
%a state machine:
%
%@example
%   static int led_init(void *unused)
%   {
%           regs[REG_AHBCLKCTRL] |= REG_AHBCLKCTRL_GPIO;
%           regs[REG_GPIO3DIR] |= 0xf;
%           return 0;
%   }
%
%   static void *led(void *arg)
%   {
%           int value, state = (int)arg;
%
%           if (state > 4)
%                   state = 0;
%           switch (state) {
%           case 4:
%                   value = 0; /* all off */
%                   break;
%           default:
%                   value = 1 << state;
%                   break;
%           }
%           regs[REG_GPIO3DAT] =  0xf & ~value;
%           return (void *)(state + 1);
%   }
%
%   static struct thos_task __task t_led = {
%           .name = "leds", .period = HZ / 5,
%           .init = led_init, .job = led,
%           .release = 10
%   };
%@end example
%
%Needless to say, it works as expected, concurrently with the tasks
%that print messages to the serial port.
%
%@c ==========================================================================
%@node The Buzzer
%@section The Buzzer
%
%The two hours of my lesson are now over, but there is no strict time
%limit in the written document. Thus, I'd like to introduce a new task,
%one that plays a tune on a buzzer. The new task is not adding anything
%to the OS core, it's just a new task in the set.
%
%@c ==========================================================================
%@c --------------------------------------------------------------------------
%@node Buzzer Hardware
%@subsection Buzzer Hardware
%
%The hardware being used for playing music is a common buzzer or
%loudspeaker (one of those you can steal from a dead PC). It is
%connected to ground and the P0_1, where one of the PWM signals
%of timer 0 can be output.
%
%@center @image{speaker, 8cm}
%
%@c ==========================================================================
%@c --------------------------------------------------------------------------
%@node Defining the Tune
%@subsection Defining the Tune
%
%
%The patch called ``The Buzzer'' in the repository adds a new file,
%@code{task-pwm.c} and augments @code{hw.h} with the register names
%we need to setup the counter 0. No other changes are needed.
%
%The task is designed as a state machine referencing a global
%@code{tune} variable. The variable is a text string naming
%the notes and pauses; the current state of the state machine
%is a pointer to the next character in the string. When at end-of-string
%the pointer is reset to the beginning of the @code{tune} string.
%Note that this is suboptimal: the task code
%itself can run two state machines concurrently because @code{tune}
%is global. The correct fix would be in defining a structure with
%two fields as task status: one field should be the tune strings for
%the task and the other a running pointer to the current note.
%
%The tune being played is the following string, which I suspect is
%a blatant violation of copyright on my side. Thus, sharing and
%distributing Thos in its current version is a crime, you are warned.
%
%@example
%   static char tune[] =
%           "f f f a c c ccc d d d b ccc aaa "
%           "f f f f a a a a g g g g fffff   "
%           "                                ";
%@end example
%
%@c ==========================================================================
%@c --------------------------------------------------------------------------
%@node Defining the Notes
%@subsection Defining the Notes
%
%The notes themselves are identified by their frequency: knowing that
%a semitone is the twelfth root of two and A is exactly 440Hz, this is
%the definition of our notes:
%
%@example
%   #define HALF 1.05946309435929526455 /* exp(2, 1/12) */
%   #define TONE (HALF*HALF)
%
%   #define F    (G/TONE)
%   #define G    (A/TONE)
%   #define A    440.0
%   #define B    (A*HALF) /* moll */
%   #define C    (B*TONE)
%   #define D    (C*TONE)
%@end example
%
%Each note is output as a square wave. The counter is programmed to run
%a 4-cycles loop with a 50% duty cycle; the actual frequency is set by
%changing the prescaler register to reach the desired output frequency.
%Values to be written in the prescaler are stored in a table, built
%at compile time from the constant frequencies we defined earlier:
%
%@example
%   struct note {
%           char name;
%           int period;
%   };
%
%   #define PWM_FREQ (THOS_QUARTZ / 4) /* we make a 4-cycles-long pwm */
%
%   struct note table[] = {
%           {'f', (PWM_FREQ/F) + 0.5},
%           {'g', (PWM_FREQ/G) + 0.5},
%           {'a', (PWM_FREQ/A) + 0.5},
%           {'b', (PWM_FREQ/B) + 0.5},
%           {'c', (PWM_FREQ/C) + 0.5},
%           {'d', (PWM_FREQ/D) + 0.5},
%           {0, ~0}, /* pause */
%   };
%@end example
%
%Our processor has no floating point unit, but the compiler converts
%the frequency values to integer when assigning the @code{period} field
%at compile time.
%
%@c ==========================================================================
%@c --------------------------------------------------------------------------
%@node The PWM Task
%@subsection The PWM Task
%
%Based on the table just shown and the tune string, this is the code of
%the task that plays the tune (I won't show the boring @code{pwm_init}
%function, which is part of the released code anyways):
%
%@example
%   static void *pwm(void *arg)
%   {
%           char *s = arg;
%           struct note *n = table;
%
%           if (!s || !*s) s = tune; /* lazy */
%
%           /* look for freq */
%           for (n = table; n->name && n->name != *s; n++)
%                   ;
%
%           /* activate it by writing the prescaler and resetting the timer */
%           regs[REG_TMR32B0PR] = n->period;
%           regs[REG_TMR32B0TCR] = 3;
%           regs[REG_TMR32B0TCR] = 1;
%
%           return s + 1;
%
%   }
%@end example
%
%At the beginning of operation (when @code{arg} is null) and at end of
%string the @code{s} variable is reset to the beginning of our tune;
%then the table is looked up, and the last entry is used for every
%unkwnown note (the blank character thus count as pause, like any other
%non-note ASCII value).
%
%The task is then defined as usual, running at a frequency of @code{HZ
%/ 10}.
%
%@c ==========================================================================
%@node Thos on ARM-7
%@section Thos on ARM-7
%
%I initially wrote Thos on the LPC-2104 processor, in the ``Christmas Tree''
%evaluation board.
%
%@center @image{ctree, 8cm}
%
%After writing the LPC-1343 version, described in this document, I
%ported the code back to the Christmas Tree, and put it in the
%@code{Thos-2104} subdirectory.  The code is slightly simpler, as the
%older device had less details to deal with. Besides, it had a serial
%port already connected, so using the UART for programming and messaging
%was definitely easier than it is now.
%
%The main differences are in register names and values (i.e.,
%@code{hw.h}). Other dissimilarities are as follows:
%
%@itemize
%@item The peripheral clock is 1/4 of the quartz frequency.
%@item There is no need to turn on the peripherals, as the are all on by default.
%@item The @code{CFLAGS} have no arch-specific flags.
%@item The buzzer has slightly different code.
%@end itemize
%
%The last item depends on the different kind of device being used: the
%buzzer you find on modern PC cases is a replacement for a loudspeaker,
%so you drive it with a square wave of the desired frequency.  The one
%mounted on the Christmas Tree is self-oscillating, and ticks at a few
%kilohertz when powered on -- for this reason the code is making a short
%positive pulse in the PWM output, changing the overall frequency.
%
%As for programming, I ported the @code{tools/} programs back to ARM-7
%where they were originally born. The current version works with both
%chip families, and can write either to RAM or to Flash memory, even
%though the code shown here only works from RAM -- the LPC-21xx family
%has no USB-storage capabilities.
%
%@c ==========================================================================
%@node The Future of Thos
%@section The Future of Thos
%
%In 2011, after writing this document, I felt I was almost done with
%Thos, unless there were bugs to fix (like a I fixed a major typo in
%original code after release 2011-04, and a student fixed some typos
%and submitted a patch).
%
%Back then I planned to port tot the Arduino (something which finally
%happened, after I started Bathos).
%But what actually happened is that I added some features and
%functionalities, that doesn't fit in the 2 hours at all, much less
%than the PWM task described earlier.  The material is libraries
%(e.g. printf), drivers (such as gpio and spi) and tasks (a shell on
%the serial port, real I/O and possibly more).
%
%While adding such code, I noticed how the simple code layout chosen
%here doesn't fit any more, even though the basic design remains the
%same and the main function (the scheduler) is unchanged.  Initially I
%tried to bring on the master branch of this repository, but after a
%few days I realized it really doesn't work because there I'm
%interested more in the final result than in having a bugless history
%to be described commit by commit.  So the new project, called
%Bathos (Born Again Thos), has a new repository, that continues
%development (and documentation) starting from release 2012-12 of Thos.
%
%The links to the repositories are as follows: they are host both on
%gitorious and on my own system.
%
%@itemize @bullet
%@item @url{git://gitorious.org/rubi/thos.git}
%@item @url{git://gitorious.org/rubi/bathos.git}
%@item @url{git://gnudd.com/thos.git}
%@item @url{git://gnudd.com/bathos.git}
%@end itemize
%
%Hope you enjoyed this, and you'll get the incentive to write your own
%OS sooner or later. In the 20th century real people didn't call
%themselves programmers until they wrote their own editor; in this
%millennium writing an OS is easier than writing an editor, so you have
%no excuse -- even if writing an HTTP server is still easier, if you can
%count on a working TCP implementation.
%
%@c ##########################################################################
%@node Bathos Design Choices
%@chapter Bathos Design Choices
%
%In order to add more features and be able to achieve real tasks with
%the simple thing described so far, we need a different code layout.
%An operating system cannot really be replicated with minor differences
%for each and every board or CPU flavour it is ported to.
%
%For this reason, the transition from Thos to Bathos is marked by a
%complete reshuffling of source code.  Now Bathos supports several
%architectures (using @t{arch-} subdirectories) with very little
%code duplication.
%
%This chapter introduces the basic design choices that I try to
%consistently follow.
%
%@c ==========================================================================
%@node Building
%@section Building
%
%At the end of November 2013 I added Kconfig to select some of
%the features. Over time I plan to move all setup to Kconfig, but
%currently the task-set is still selected on the command line.
%
%To build a binary system, you should perform the following steps:
%
%@itemize @bullet
%
%@item Configure the architecture, by running @t{make menuconfig}
%or equivalent command. @t{make defconfig} selects the ram-based
%build for LPC1343.
%
%@item Choose your cross-compiler. Please note that the environment
%variable @t{CROSS_COMPILE} is still supported, even if you can
%specify the compiler prefix in the configuration.
%
%@item Run @t{make}.  The chosen task-set is the default for
%the chosen architecture.  To select a different task-set you
%can pass on the command line a @i{make} variable:
%@t{TASK-y=}. By using the @t{-y} convention we can use configuration
%variables like @t{CONFIG_XXX}, which are being moved to the Kconfig
%system.
%
%@end itemize
%
%@c ==========================================================================
%@node Git History
%@section Git History
%
%The main aim of this project is simplicity. This, for me, includes
%helping my readers in understanding the choices I made; thus I tend
%to be quite self-disciplined about the repository history, because
%it's a great supplement to documentation -- I noted that in serious
%projects I follow or contribute to, and I want to do the same for
%my users.
%
%Every major set of changes is marked in @i{git} history as the merge of a
%branch with a meaningful name, like ``file-layout'' or ``flashing''.
%This is
%done to help my readers in seeing how change sets are grouped.
%Usually, in such branches, the last commit
%before the merge is documentation for the whole set of commits, so
%the merge points are bot stable and documented.  The published @i{master}
%branch is always pushed at such stable points, but other branches are
%available for those who want to see work in progress.
%
%Unfortunately,
%it may happen that some details are fixed or added later; in that case
%I won't back-port the commit to the original branch, in order not to
%change past history of the @i{master} branch.
%
%@c ==========================================================================
%@node Initcalls
%@section Initcalls
%
%Bathos supports initcalls, in the typical Linux way.  This means that
%most initialization functions are marked as @t{core_initcall} or
%@t{task_initcall} (see @t{<bathos/init.h>} for the full set of names.
%
%Pointers to functions declared as @i{initcalls} are placed in a
%special ELF section, where initialization code finds them, sorted by
%initialization level.
%
%The @t{rom_initcall} declaration is slightly different from all other
%initcalls: the pointer is only used if you build for flash. This
%allows, for example, to avoid initializing the serial port in the
%@sc{lpc1343} if the code is being programmed through that very
%serial port; in that case the internal SoC ROM already configured
%the port and we'd better avoid doing it again.
%
%@c ==========================================================================
%@node Kconfig
%@section Kconfig
%
%Configuration of the architecture and the memory mode is currently
%performed through Kconfig. The code base I imported is from
%Linux-2.6.34, and while I tried a forward port to Linux-3.4 code, the
%result is not worth the big amount of code lines, in my opinion.
%
%As usual, you can run ``@t{make config}'' and ll the other variants:
%@i{menuconfig}, @i{xconfig}, @i{gconfig}. For each you need the related
%development libraries (@i{ncurses}, @i{kde} and @i{gnome}, resp.).
%
%Running @t{make defconfig} selects @sc{lpc1343} in @sc{ram} mode, like
%the original Thos.  Each architecture and memory mode has a pre-build
%configuration in @t{./configs}, so for example``@t{make
%unix_defconfig}'' selects a Unix build.
%
%The configurations I provide include a selection of @t{CROSS_COMPILE},
%but the environment variable takes precedence, if set.  I lazily provided
%my own configuration in the default configs, so the @t{./MAKEALL} script
%can build for all architectures without user interaction.  I welcome
%patches to make @t{MAKEALL} more independent of my own setup.
%
%@c ==========================================================================
%@node Linking Object Files
%@section Linking Object Files
%
%While building Bathos, I first create a big object file, called
%@t{bathos.o}, that includes everything that is expected to be part of
%the final binary.  The rule to create @t{bathos.o} references the
%libraries and everything needed.  The final link step is then
%performed using an architecture-specific and memory-mode-specific
%linker script.
%
%The step through @t{bathos.o} allows me to implement initcalls in a
%single place, i.e. @t{bigobj.lds}, the linker script used to build the
%intermediate object file.  The alternative, to avoid code duplication,
%would be preprocessing the final linker script, to include initcall
%management in each and every architecture.  I currently prefer the
%intermediate object file for this project, to keep the final linker
%scripts simpler than they would.
%
%
%@c ==========================================================================
%@node Other Points about Code
%@section Other Points about Code
%
%If you want to play with the code base, there a few more points
%that you should be aware of:
%
%@itemize @bullet
%
%@item Most @i{make} variables are built incrementally.
%
%@item Bathos-specific headers live in @i{include/bathos} and
%are referenced as @t{<bathos/xxx.h>} by source files.
%
%@item Architecture-specific files live in a directory called @t{arch-sth}.
%Such files are usually @t{io.c}, assembly code, the linker script
%and a @t{Makefile} for arch-specific settings, like special @t{CFLAGS}. See
%available files for examples.
%
%@item Some architectures have two linker scripts, for ram-based and
%flash-based build. The name being used is selected by @t{CONFIG_MEMORY_MODE}.
%
%@item Portable tasks live in @t{tasks/}
%
%@item Architecture-specific tasks live in @t{tasks-sth} (for the
%architecture @t{sth}). For example, the LED and PWM tasks in the
%two Thos ports rely on specific bit numbers (until I put such configuration
%in Kconfig variables).
%
%@item Other subdirectories, like @t{lib/}, have the usual meaning.
%
%@end itemize
%
%@c ##########################################################################
%@node Supported Architectures 
%@chapter Supported Architectures
%
%The current version of Bathos supports several architectures, and a
%few of them can be built in two modes (@i{ram} and @i{flash}).
%
%The architecture is selected by running the configurator,
%as described in @ref{Bathos Design Choices}. You can still pass
%@t{ARCH=} on the command line of @i{make} or in the environment,
%but this is not the suggested way to configure Bathos.
%
%When more than one mode is supported (i.e., both RAM and Flash
%can be used to run the program), the configurator
%allows to choose one but, again, you can pass @t{MODE=} on the
%command line.
%@table @code
%
%@item ARCH=unix
%
%	The Unix build runs Bathos as a process.  This setup is the
%        most easily tested, as it runs
%        on the build host. It currently only supports the serial tasks.
%
%@item ARCH=lpc1343
%
%	This is the default (i.e., @t{make defconfig}), and by
%        defuautl selects a RAM build, to compile
%        the original Thos code for the respective hardware device
%        (see @ref{Introduction to Thos}).  The port includes all the
%        tasks described earlier.  @i{lpc1343} can be built for ram or flash.
%
%@item ARCH=lpc2104
%
%	Compile for the ARM-7 processor, LPC2104 (or 2105 or 2106).
%        This architecture carries hardware support for the @i{Christmas Tree}
%        (see @ref{Thos on ARM-7}). The port includes all the tasks
%        and can be built for ram or flash.
%
%@item ARCH=versatile
%
%	The @i{Versatile} board is on of the evaluation boards released
%        by ARM Ltd. It is supported by @i{Qemu}, and allows to run
%        the code without the need for real hardware.  The executable
%        script @t{arch-versatile/run-bathos} can be used to actually run
%        the operating system.  The port only supports the serial
%        tasks, so please compile with ``@t{TASK-y=task-uart.o}''.
%        The build is supported for RAM only.
%
%@item ARCH=atmega
%
%	This is a port to the Arduino board. I run it on the ATmega8
%        implementation of the AVR architecture. Unfortunately, there
%        are quite some differences between the various implementations,
%        so most likely this won't work on other flavours (like the ATmega328
%        that features a different size for the interrupt vectors). The
%        port uses timer interrupts to increment @i{jiffies}, and the
%        code must be flashed to the micro-controller, that doesn't allow
%        running from RAM. The port only supports the serial
%        tasks, so please compile with ``@t{TASK-y=task-uart.o}''.
%
%@end table
%
%In the future I'll add more architectures.
%Also, I plan to support the ATmega328 without
%introducing a new architecture for it.
%
%@c ##########################################################################
%@node Printf Function
%@chapter Printf Function
%
%The first feature added to Bathos is a real @i{printf} function.  But
%printing a number in ASCII requires a series of divisions; when
%hardware lacks division circuitry, the compiler may choose to make a
%function call, when it is cheaper than running inline code.  For
%example, on ARM7, dividing by 10 turns into a call to
%@t{__aeabi_idiv}.  Converting a number to a string, thus, requires
%@i{libgcc} if you compile for the 2104 architecture.  Even though the
%problem doesn't happen on the Cortex, linking @t{libgcc.a} has no side
%effects if you have no external symbols, so Bathos links in @t{libgcc}
%in the top-level Makefile.
%
%The directory @i{pp_printf} includes a copy of the @i{Poor
%Programmer's printf}, that I wrote a few years ago and pushed to
%@i{gitorious} in 2012 after cleaning up and documenting a little.
%See the commit log for details.
%
%This implementation of @i{printf} relies on @i{puts}, as described
%earlier (in Thos), and comes in 4 flavours, differing in features
%supported and code size.  The full-featured one I picked from
%@i{U-Boot}, and it actually comes from an older Linux kernel; the
%smaller implementations are my own hacks, removing or replacing code
%from the full one. See @t{pp_printf/README} for details.
%
%The flavour I selected to compile for Thos is the @i{xint} one, which only
%supports hex and integers. It's a good compromise between features
%and space. It is used in @i{main.c} to simplify a hairy
%line, but it is mainly is a tool that can be used by later code.
%
%Finally, the linker script uses @t{PROVIDE()} to offer a @i{printf}
%name as an alias for @i{pp_printf}. I might have used the preprocessor
%instead, but I prefer to leave the @i{pp-printf} package unchanged.
%
%@c ##########################################################################
%@node GPIO Support
%@chapter GPIO Support
%
%Bathos offers a simple GPIO abstraction. The code for LPC1343 comes
%from another project of mine, while code for LPC2104 (which is
%much simpler) has been written anew for Bathos. 
%
%@c ==========================================================================
%@node The GPIO API
%@section The GPIO API
%
%This is the programming interface I'm using for GPIO. The code
%lives partly in @i{gpio.h} and partly in @i{gpio.c}. Each architecture
%has its own files, if it support GPIO, but the API is the same.
%
%The LED tasks have been modified to use this API.
%
%@table @code
%@item GPIO_NR(port, bit);
%@itemx GPIO_PORT(nr);
%@itemx GPIO_BIT(nr);
%	These macros convert from port+bit to number and back. They
%	are not expected to be used often, but they may be useful.
%        The port number runs from 0 to the number of device ports, and
%        the bit number runs from 0 to 31 (or to 11, or other values,
%        according to the specific hardware design.
%
%@item void gpio_init(void);
%	Initialization is needed on some architectures, because
%        the GPIO and pin-connect clocks
%        of the chip may need to be powered for the following functions to work.
%        Even though here initialization of the UART is already doing
%        the required setup in some cases,
%        it's in general good practice to have an
%        init function and call it before using the module.
%
%@item int gpio_dir_af(int gpio, int output, int value, int afnum);
%@itemx int gpio_dir(int gpio, int output, int value);
%	The former function sets the alternate function for a bit, and it
%	configures it as input or output. We have macros for the values,
%        see below.
%	The latter function only changes the direction (this is useful
%	since changing the mode is much more costly in term of machine
%	instructions. The @i{value} argument is needed to change the
%	GPIO output bit while changing the mode. Some devices allow
%        to set the output value first, and some require to set it
%        after switching the mode; each implementation is responsible
%        to do the right thing, which is invisible to the user.
%
%@item int gpio_get(int gpio);
%@itemx u32 __gpio_get(int gpio);
%	The functions return the current input value. The former
%        return 0 or 1, while the latter returns 0 or non-0. As usual
%	in several context, the double underscore means the function is
%        ``internal'' or ``lower level''.
%
%@item void gpio_set(int gpio, int value);
%@item void __gpio_set(int gpio, u32 value);
%	The functions set an output bit. The former receives 0 or 1,
%        while the latter receives 0 or the bit value -- i.e., the same
%        value that is returned by @i{__gpio_get}, to save a few
%        instructions in some common situations, 
%@end table
%
%The following macros are offered, because I used to forget the meaning
%of the numbers. The idea is that code should be self-documenting, so
%you should use those names and ease your reader (or yourself in the
%future).
%
%@table @code
%@item GPIO_DIR_IN
%@item GPIO_DIR_OUT
%
%	0 and 1, for @t{output} argument for @t{gpio_dir} and @t{gpio_dir_af}.
%
%@item GPIO_AF_GPIO
%@item GPIO_AF(value)
%
%	The former is 0; both are for @t{gpio_dir_af}.
%@end table
%
%@c ==========================================================================
%@node Design Choices behind GPIO
%@section Design Choices behind GPIO
%
%The GPIO pins in the 1343 are especially strange.  The GPIO
%abstraction is designed to hide those peculiarities, and make the
%interface as portable as possible.  This section describes the choices
%I made for the LPC 1343 (the LPC 2104 is much simpler and no brain is
%needed to make its GPIO interface). The uninterested reader may
%skip the whole section.
%
%The choices described here are not the only way to abstract GPIO,
%and you may agree or disagree with what I did.
%
%For performance, the @i{gpio} argument is
%checked for correctness only in the configuration function, which is
%expected to be executed at least once before using the bit, and possibly
%only once in the life of your program.
%
%As a prerequisite, the code introduces the standard functions
%@i{readl} and @i{writel}. This is a different I/O abstraction
%than the @t{regs} array used in Thos and Bathos, but this
%file comes from another project, and I retained the code that
%was already debugged there. @i{readl}/@i{writel} is not worse
%than @t{regs}, it's just a different way to achieve the same
%technical effect.  Similarly, the code uses a few types that
%are not standard but very handy, the most important being @code{u32}.
%
%@subsubheading GPIO Numbering
%
%Most other processors I work with have their GPIO pins divided in
%``ports'', each port being 32 bits wide.  Both ports and bits are
%counted starting from 0 -- sometimes ports start from A and use
%alphabetic letters.
%
%In the operating systems I use daily, gpio numbers are just
%numbers, whether or not the hardware docs talk about ports or not.  So
%there bit 10 of port 2 is GPIO 74 (2 * 32 + 10), irrespective of how
%hardware is.  Here I use the same approach; it eases development of
%generic device drivers, like a bit-banged I2C driver based on two GPIO
%pins.
%
%As a practical result, on the LPC1343 (which has 12 bits per port)
%we can use GPIO 0 through 11, 32 through 43 and
%so on.  Macros to convert to and from port-and-bit are provided
%nonetheless, as they are hardware-independent conversions.
%
%@subsubheading Alternate Functions
%
%Most processors have alternate functions, and GPIO is usually function
%0. Here we have up to 7 ``alternate functions'' and one GPIO function.
%Then the 1343 has other features (like hysteresis) that I'm not supporting
%by now.
%
%The LPC13 device is set up strangely, in that the PIO function is not
%always function 0 -- sometimes it is function 1.  But, for the sake
%of portability, we need function 0 to always represent PIO.
%
%For example, a generic LED or key driver (or bit-bang I2C or whatever)
%needs to configure its own bits as PIO, irrespective of what the host
%hardware is (most likely, the bit numbers come from a data structure,
%so the driver ignores how to configure them).  For this reason,
%@code{AF0} is always the PIO function. I define bit masks in order for
%the GPIO code to swap AF0 with AF1 for those bits where this is
%needed.
%
%@subsubheading Accessing Configuration Registers
%
%Even though the GPIO configuration registers are all alike, their
%placement in the memory map is absolutely random for the 1343.  Here the
%need is describing this placement in the smallest possible space.
%
%Please note that configuration registers are used rarely, because
%the PIO operations (including switching the direction of one bit)
%are performed on different registers.
%
%To keep the code as small as possible, I define an array of offsets,
%one per GPIO bit, stating where the relevant register lives in the
%associated memory area.  Such offsets are 8 bits long, and are
%the index of the register, so they are shifted by two bits before
%being added to the base register. This saves storage space
%in exchange for some calculation, but as said pins are configured
%usually once for the whole uptime of the system.
%
%In my experience, people looking at documentation are used to look for
%relevant symbolic names in the headers, grepping for the hex
%address. For this reason, my header prefers compile-time calculation
%to build the offsets starting from the complete hex number.  Moreover,
%this saves users from checking and rechecking the table when looking
%for other bugs.
%
%@subsubheading Changing several bits at the same time
%
%Although the LPC13 allows to atomically change an arbitrary
%set of bits as long as they are part of the same port, I offer
%no support for this at API level -- I plan to add it later, but
%it's not there yet.
%
%@c ##########################################################################
%@node SPI Support
%@chapter SPI Support
%
%SPI, ``Serial Peripheral Interface'' is a simple synchronous bus, used
%by a number of peripherals: ADC and DAC devices, display controllers,
%simple network adapters and so on. In its most common form it is made
%up of 4 wires: clock, chip select and two data lines: MOSI (master out
%slave in) and MISO (master in slave out).  You can connect more than
%one peripheral to the same bus: each device has a personal CS line,
%while the other three wires are shared.
%
%SPI code in Bathos is, again, derived from a previous project. It is
%currently only supported for the LPC2104 architecture, because that is
%what I needed. A port to LPC1343 is ongoing, and Atmega should follow.
%
%In theory you can make a bit-banged SPI controller, based on GPIO pins,
%but in practice SPI is usually devoted to fast peripherals and thus is
%driven by hardware controllers.  There is currently no support for
%bit-banged SPI controllers in Bathos.
%
%@c ==========================================================================
%@node The SPI API
%@section The SPI API
%
%The abstraction being used is based on a few data structures and a
%few functions. The data structures are ``configuration'', ``device''
%and the data buffers:
%
%@table @code
%
%@item struct spi_cfg
%
%	The configuration structure defines which GPIO pin is used
%        as chip-select for this device, the frequency (currently unused),
%        polarity and phase (see SPI documentation) and the ``controller''
%        index, that can be used to select among several devices in the
%        microcontrollers that include more than one such controller.
%
%@item struct spi_dev
%
%	The device includes a pointer to the current configuration and
%        other fields, that are private to the implementation.
%
%@item struct spi_ibuf
%@itemx struct spi_obuf
%
%	The input and output buffers are made up of a length and an
%        array of bytes. The output buffer differs by hosting a @t{const}
%        array (so it can remain in flash memory, when the task is so
%        designed).
%
%@end table
%
%Bathos offers the following functions to deal with SPI:
%
%@table @code
%
%@item struct spi_dev *spi_create(struct spi_dev *dev);
%@itemx void spi_destroy(struct spi_dev *dev);
%
%	A device must be created before being used. Creation, however,
%        doesn't involve any allocation as we have no @t{malloc} in Bathos,
%        as a design choice to avoid run-time memory shortage.  The caller
%        must pass a pointer to its own static copy of @t{spi_dev}, which
%        is returned back after initialization.  The @t{destroy} function
%        is provided for symmetry, but typical use cases don't involve
%        calling it.
%
%@item int spi_xfer(dev, flags, ibuf, obuf);
%
%	Perform an SPI transfer. Either @t{ibuf} or @t{obuf} can be NULL,
%        it the caller is only interested in one of the two directions.
%        By passing flags, the caller can avoid activating or releasing
%        the chip select pin, or both; this allows supporting longer
%        transfers without producing edges on the chip select, which is
%        requested by some devices that use it as a reset or low-power
%        signal.
%
%@item  int spi_read(dev, flags, ibuf);
%@itemx int spi_write(dev, flags, ouf);
%
%       The two functions are shortcuts that call @t{spi_xfer}
%       with a NULL pointer for the unused buffer.  Developers are encouraged
%       to use them to make the code more readable, according ot the use case.
%
%@item int spi_raw_read(dev, flags, len, uint8_t *buf);
%@itemx int spi_raw_write(dev, flags, len, uint8_t *buf);
%
%      Read or write a raw buffer, without relying on @t{struct spi_ibuf}
%      or @t{struct spi_obuf}.  They should be used according to the
%      use case.
%
%@end table
%
%Please note that the 4 read and write functions are implemented as
%@t{static inline}, and thus are platform-independent.  The device
%driver for the target architecture only implements @t{spi_xfer}.
%
%@c ==========================================================================
%@node SPI Sample Tasks
%@section SPI Sample Tasks
%
%Bathos includes a few tasks that act on SPI devices:
%
%@table @code
%
%@item task-spi
%
%	This is the first, simple SPI demonstration task. It drives an
%        AD7888 8-channel ADC, reporting the conversions as 8 hex numbers
%	using @i{printf}. I uses GPIO9 as chip select pin, you may need
%        to change it according to your personal configuration.
%
%@item task-enc28
%
%	This task sends and receives network frames using the EN28J60
%        integrated circuit.  It relies on @t{drivers/enc28j60.c}, which
%        in turn relies on the SPI API.  It exists mainly for demonstration
%        purposes, but if you manage to run it you know your network
%        is working and can build on it.
%
%@end table
%
%@c ##########################################################################
%@node OneWire
%@chapter OneWire
%
%The OneWire protocol is supported by a set of files called @t{w1-*},
%and the API is defined in @t{<bathos/w1.h>}.  Currently the only
%available backend is @t{w1-gpio.c}, which in turn relies on
%@i{udelay}.
%
%@c ==========================================================================
%@node W1 API
%@section W1 API
%
%The code is based on the context of @i{bus}, and relies on some
%hardware-specific methods, host in @t{struct w1_ops}.  Due to my
%laziness, currently there can only be one set of operations: while I
%support several busses of the same kind (i.e., several wires), you
%can't define busses of different kinds (e.g., one using GPIO and the
%other using an hardware controller.
%
%These are the components of the API, defined in @t{<bathos/w1.h}:
%
%@table @code
%
%@item struct w1_bus
%
%	A bus can host at most 8 devices, because the structure includes
%        a static array of devices. It also includes a @t{detail} field
%        that can be used to differentiate between different busses. For
%        the GPIO back-end, the detail is the GPIO number of this wire.
%
%@item struct w1_dev
%
%	The device is found by library functions, and tasks can access
%        the array of devices in @t{bus->devs}.
%
%@item int w1_scan_bus(struct w1_bus *bus);
%
%	The function is used to scan a new bus and enumerate the
%        devices in there. The caller is expected to fill the
%        @t{detail} field beforehand.
%
%@item void w1_write_byte(struct w1_bus *bus, int byte);
%@itemx int w1_read_byte(struct w1_bus *bus);
%
%	Low-level operations to communicate on the bus.
%
%@item void w1_match_rom(struct w1_dev *dev);
%
%	This higher-level function issues the ``@i{match rom} command
%        for the specific device. The caller can use it to select one
%        of the devices on the bus and then send and receive bytes
%        from it, even if other devices exist on the bus.  The device
%        will remain selected until @t{w1_match_rom} is called again,
%        for a different device, or @t{w1_scan bus} is called again.
%
%@end table
%
%Since devices are usually only thermometers or EEPROM devices, the
%library offers specific support for some of those devices: the
%thermometer DS18S20, DS18B20 and DS28EA00, and the EEPROM DS28EC20
%(class 0x43).  I accept patches for other families.
%
%The following functions work on the specific devices:
%
%@table @code
%
%@item int32_t w1_read_temp(struct w1_dev *dev, unsigned long flags);
%@item int32_t w1_read_temp_bus(struct w1_bus *bus, unsigned long flags);
%
%	The former function reds the temperature from a specific
%        device; the latter uses the first thermometer in your bus,
%        according to the result of the initial bus scan. Temperature
%        is returned as a fixed point floating number, with 16 integer
%        bits and 16 fractional bits. If an error occurs, the functions
%        return  0x8000.0000.
%
%        The flags argument can be used to issue the command without
%        waiting for the result, or rather collect the result of a previous
%        read-temperature command.  When passing 0, the code waits for
%        temperature to be converted (i.e., it takrs more or less 0.5s).
%
%@item int w1_read_eeprom(struct w1_dev *dev, int offset, uint8_t *buffer, int blen);
%@itemx int w1_read_eeprom_bus(struct w1_bus *bus, int offset, uint8_t *buffer, int blen);
%
%      The functions read an area of the EEPROM; either from the specified
%      device of from the first device of class 0x43. The functions return
%      the number of bytes successfully read, or a negative number in case
%      of error.
%
%@item int w1_write_eeprom(struct w1_dev *dev, int offset, const uint8_t *buffer, int blen);
%
%@itemx int w1_write_eeprom_bus(struct w1_bus *bus, int offset, const uint8_t *buffer, int blen);
%
%      The functions write an area of the EEPROM; either from the specified
%      device of from the first device of class 0x43. The functions return
%      the number of bytes successfully written, or a negative number in case
%      of error.
%
%@end table
%
%@c ==========================================================================
%@node W1 Sample Task
%@section W1 Sample Task
%
%The sample task @t{task-w1.c} can be used to see how to use the W1 API.
%It accesses GPIO 8 to communicate with a bus, and it can be built with:
%
%@example
%   make TASK-y=task-w1.o
%@end example
%
%This is what happens in a system with three devices: the task uses
%@t{w1_read_temp_bus} and @t{w1_read_eeprom_bus}, so it accesses the
%``first'' suitable device. Thus, it ignores the fact that there are
%two thermometers.
%
%@example
%   Task: w1
%   w1_init: scan result: 3 devices
%     68000801dce56910
%     f70000001eda8242
%     5f00000040e50143
%   temp 0x0013e000: 19.8750
%   eeprom data @ 0x00000000: 53 44 42 2d 00 05 01 01
%   temp 0x0013c000: 19.7500
%   eeprom data @ 0x00000008: 00 00 00 00 00 00 00 00
%   temp 0x0013d000: 19.8125
%   eeprom data @ 0x00000010: 00 00 00 00 00 00 03 7f
%   temp 0x0013f000: 19.9375
%   eeprom data @ 0x00000018: 46 69 6c 65 44 61 74 61
%   temp 0x00135000: 19.3125
%   eeprom data @ 0x00000020: 2e 20 20 20 00 00 00 01
%@end example
%
%The task acts on each device every 4 seconds, and reports
%no error if no device is found; you can see that there is no device
%from the initial @t{scan result} message.
%
%While I could re-scan the bus at each iteration, this task does it only
%at startup, in its own @i{init} function; thus, if you replace the device
%after power on, the new device won't be used.
%
%
%@c ##########################################################################
%@node Drivers
%@chapter Drivers
%
%The directory called @t{drivers} includes arch-independent code to
%deal with specific integrated circuits or software protocols. For example,
%the OneWire (@t{w1}) protocol and its hardware-independent gpio backend
%lives here. Drivers will host the I2C code
%the bit-banged I2C driver, while arch-specific
%code using hardware I2C controllers lives in the respective @t{arch}
%directory.
%
%This is the current list of drivers:
%
%@table @code
%
%@item enc28j60
%
%	The chip is a 10Mb Ethernet controller, connected through SPI.
%        The driver offers @i{create} and @i{destroy} functions, so the
%        specific application can drive more than one device, if
%        needed.  The header @t{<bathos/enc28j60.h>} should be enough
%        to understand how this works, with the help of the sample task
%        @t{tasks/task-enc28.c}.
%
%@item OneWire
%
%	See @ref{OneWire}.
%
%@item lcd44780
%
%	This is the classical LCD display, typically sold as 16x2 ASCII
%        characters, connected on a 4-bit data bus, plus control signals.
%        The driver currently only supports 4-bit mode.  The user must
%        tell which GPIO lines are connected to the display. Please see
%        @t{tests/test-lcd44780.c} for an example use.
%
%@item dht11
%
%	This is a temperature and humidity sensor. It uses a custom
%        protocol over a single wire, but this means reading requires
%        a whole FIXME milliseconds, using 100% CPU time.  Also, it requires
%        it's own dedicated GPIO wire. The driver is set up as an init
%        and a job function, ready to be inteserted in a Bathos
%        task. As an example use, please see @t{tasks/task-dht11.c}.
%
%@end table
%
%@c ##########################################################################
%@node Lib
%@chapter Lib
%
%The @t{lib} directory hosts a number of standard or generic functions.
%The object files are linked into @t{libbathos.a}.
%
%This is a list of source files and their role:
%
%@table @code
%
%@item setup.c
%
%	This implements @i{bathos_setup} using the @i{initcall} mechanism.
%        The function is @i{weak}, so the linker will pick an arch-specific
%        instance of the function if it exists, using this as a fall-back
%        implementation.
%
%@item udelay.c
%
%	The file includes a generic @t{udelay} function, that runs a
%        busy loop for the requested number of microseconds.  The function
%        is calibrated at system startup (using an @i{initcall}) by
%        calculating the @t{lpj} value, or @i{loops per jiffies}. The
%        delay itself is then achieved by running the same loop, so the
%        precision of the delay is acceptable.  There clearly is an
%        overhead in this function, so there is a constant added latency,
%        which depends on the architecture.
%        @c FIXME: find and list the latency
%
%@item ctype.c
%@itemx stdio.c
%@itemx string.c
%
%	Standard functions people may want to use in the code, including
%        @t{memset} that is called by the compiler itself. The functions
%        here are a subset of the ones you find in @i{libc}, so for
%        example there is no @i{strcmp} at this point and there will likely
%        never be @i{stricmp}.
%
%@end table
%
%
%
%
%@bye
%
%@c  LocalWords:  thos Olimex README rubini gnudd Alessandro titlepage texinfo
%@c  LocalWords:  iftex smallexample linux CFLAGS LDFLAGS ASFLAGS rodata objdump
%@c  LocalWords:  GPIO LocalWords objcopy putc THRE microcontroller regs readl
%@c  LocalWords:  writel init struct Arduino HTTP printf AHBCLKCTRL uart Kconfig
%@c  LocalWords:  initcall initcalls libgcc gpio
%
%
%
%\end{document}